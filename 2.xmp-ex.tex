\chapter{{\XMP} Extensions}\label{chap:xmp-ex}
This chapter defines a behavior of mixing {\XMP} and {\OACC}.
Note that the existing {\OACC} is not extended in the {\XMP} extensions.
The {\XMP} extensions can represent 
(1) parallelization with keeping sequential code image using a combination of {\XMP} and {\OACC},
and
(2) communication among accelerator memories and between accelerator memory and host memory on different {\bf nodes}
using {\XACC} directives or {\bf coarray} features.

\section{Combination of {\XMP} and {\OACC}}
\subsection{{\OACC} Directives on Data}
\subsubsection*{Description}
When {\bf distributed arrays} appear in {\OACC} constructs,
global indices in {\bf distributed arrays} are used.
%Thus,
%an {\XACC} compiler automatically translates the global indices in the {\OACC} constructs to the appropriate local indices.
The {\bf distributed arrays} may appear in the {\tt update}, {\tt enter data}, {\tt exit data}, 
{\tt host\_data}, {\tt cache}, and {\tt declare} directives,
and the {\tt data} clause accompanied by some of 
{\tt deviceptr}, {\tt present}, {\tt copy}, {\tt copyin}, 
{\tt copyout}, {\tt create}, and {\tt delete} clauses.
Data transfer of {\bf distributed array} by {\OACC} is performed on only {\bf nodes} which have elements specified by the global indices.

\subsubsection*{Example}
\begin{myfigure}
\begin{minipage}{0.45\hsize}
\begin{center}
\begin{XACCFexampleL}
integer :: a(N), b(N)
!$xmp template t(N)
!$xmp nodes p(*)
!$xmp distribute t(block) onto p
!$xmp align a(i) with t(i)
!$xmp align b(i) with t(i)
...
!$acc enter data copyin(a(1:K))
!$acc data copy(b)
...
\end{XACCFexampleL}
\end{center}
\end{minipage}
%
\begin{minipage}{0.53\hsize}
\begin{center}
\begin{XACCCexampleR}
int a[N], b[N];
#pragma xmp template t[N]
#pragma xmp nodes p[*]
#pragma xmp distribute t[block] onto p
#pragma xmp align a[i] with t[i]
#pragma xmp align b[i] with t[i]
...
#pragma acc enter data copyin(a[0:K])
#pragma acc data copy(b)
{ ...
\end{XACCCexampleR}
\end{center}
\end{minipage}
\caption{Code example in {\XMP} extensions with {\tt enter\_data} directive}\label{code:ex-oacc-data}
\end{myfigure}

In lines 2-6 of Fig. \ref{code:ex-oacc-data},
the directives declare the {\bf distributed arrays} {\it a} and {\it b}.
In line 8,
the {\tt enter data} directive transfers the certain range of the {\bf distributed array} {\it a} from host memory to accelerator memory.
Note that the range is represented by global indices.
In line 9,
the {\tt data} directive transfers the whole {\bf distributed array} {\it b} from host memory to accelerator memory.

\subsection{{\OACC} Loop Construct}
\subsubsection*{Description}
In order to perform a loop statement on accelerators in {\bf nodes} in parallel,
{\XMP} {\tt loop} directive and {\OACC} {\tt loop} directive are used.
While
{\XMP} {\tt loop} directive performs a loop statement in {\bf nodes} in parallel,
{\OACC} {\tt loop} directive also performs the loop statement parallelized by the {\XMP} {\tt loop} directive 
on accelerators in parallel.
For ease of writing,
the order of {\XMP} {\tt loop} directive and {\OACC} {\tt loop} directive does not matter.

When {\tt acc} clause appears in {\XMP} loop directive with {\tt reduction} clause,
the directive performs a reduction operation for a variable specified in the {\tt reduction} clause on accelerator memory.

\subsubsection*{Restriction}
\begin{itemize}
\item In {\OACC} {\bf compute region},
only {\XMP} {\tt loop} directive without {\tt reduction} clause can be inserted.
\item In {\OACC} {\bf compute region},
targeted loop condition (lower bound, upper bound, and step of the loop) must remain unchanged.
\item {\tt acc} clause in {\XMP} loop directive can appear only when {\tt reduction} clause appears there.
\end{itemize}

\subsubsection*{Example 1}
\begin{myfigure}
\begin{minipage}{0.45\hsize}
\begin{center}
\begin{XACCFexampleL}
integer :: a(N), b(N)
!$xmp template t(N)
!$xmp nodes p(*)
!$xmp distribute t(block) onto p
!$xmp align a(i) with t(i)
!$xmp align b(i) with t(i)
...
!$acc parallel loop copy(a, b)
!$xmp loop on t(i)
do i=0, N
  b(i) = a(i)
end do
!$acc end parallel
\end{XACCFexampleL}
\end{center}
\end{minipage}
%
\begin{minipage}{0.53\hsize}
\begin{center}
\begin{XACCCexampleR}
int a[N], b[N];
#pragma xmp template t[N]
#pragma xmp nodes p[*]
#pragma xmp distribute t[block] onto p
#pragma xmp align a[i] with t[i]
#pragma xmp align b[i] with t[i]
...
#pragma acc parallel loop copy(a, b)
#pragma xmp loop on t[i]
for(int i=0;i<N;i++){
  b[i] = a[i];
}

\end{XACCCexampleR}
\end{center}
\end{minipage}
\caption{Code example in {\XMP} extensions with {\OACC} loop construct}\label{code:ex-oacc-loop}
\end{myfigure}

In lines 2-6 of Fig. \ref{code:ex-oacc-loop},
the directives declare {\bf distributed arrays} {\it a} and {\it b}.
In line 8,
the {\tt parallel} directive with the {\tt data} clause transfers the {\bf distributed arrays} {\it a} and {\it b} from host memory to accelerator memory.
Moreover,
in lines 8-9,
the {\tt parallel} directive and {\XMP} {\tt loop} directive perform the next loop statement on accelerators in {\bf nodes} in parallel.

\subsubsection*{Example 2}
\begin{myfigure}
\begin{center}
\begin{XACCFexampleL}
integer :: a(N), sum = 10
!$xmp template t(N)
!$xmp nodes p(*)
!$xmp distribute t(block) onto p
!$xmp align a(i) with t(i)
...
!$acc parallel loop copy(a, sum) reduction(+:sum)
!$xmp loop on t(i) reduction(+:sum) acc
do i=0, N
  sum = sum + a(i)
end do
!$acc end parallel loop
\end{XACCFexampleL}
\begin{XACCCexampleL}
int a[N], sum = 10;
#pragma xmp template t[N]
#pragma xmp nodes p[*]
#pragma xmp distribute t[block] onto p
#pragma xmp align a[i] with t[i]
...
#pragma acc parallel loop copy(a, sum) reduction(+:sum)
#pragma xmp loop on t[i] reduction(+:sum) acc
for(int i=0;i<N;i++){
  sum += a[i];
}
\end{XACCCexampleL}
\end{center}
\caption{Code example in {\XMP} extensions with {\OACC} loop construct with reduction clause}\label{code:ex-oacc-loop-reduction}
\end{myfigure}

In lines 2-5 of Fig. \ref{code:ex-oacc-loop-reduction},
the directives declare {\bf distributed array} {\it a}.
In line 7,
the {\tt parallel} directive with the {\tt data} clause transfers the {\bf distributed array} {\it a} and variable {\bf sum} from host memory to accelerator memory.
Moreover,
in lines 7-8,
the {\tt parallel} directive and {\XMP} {\tt loop} directive perform the next loop statement on accelerators in {\bf nodes} in parallel.
When finishing the calculation of the loop statement,
OpenACC {\tt reduction} clause and {\XMP} {\tt reduction} and {\tt acc} clauses in lines 7-8 perform a reduction operation for the variable {\bf sum} on accelerators in {\bf nodes}.

\section{Communication on Accelerated Clusters}
\subsection{{\XACC} Directives}
{\XACC} directives are extensions of {\tt reflect}, {\tt gmove}, 
{\tt barrier}, {\tt reduction}, {\tt bcast}, and {\tt wait\_async} directives in {\XMP} global-view memory model.
Moreover,
{\tt reflect\_init} and {\tt reflect\_do} directives are added as extensions of the {\tt reflect} directive.
{\XACC} directives are directives which are added an {\tt acc} clause to the above directives.
{\XACC} directives transfer data stored on accelerator memory.
Note that while XcalableACC {\tt gmove} directive described in Section \ref{sec:reflect} 
and {\bf coarray} features described in Section \ref{sec:coarray} can perform communication both among accelerator memories and between accelerator memory and host memory on different {\bf nodes},
other directives can perform communication only among accelerator memories.

This section describes only the extended parts of {\XACC} directives from {\XMP} directives. 
For other information, refer to the {\XMP} specification\cite{xmp}.

\subsubsection{reflect Construct}\label{sec:reflect}
\subsubsection*{Synopsis}
The {\tt reflect} construct assigns the value of a
reflection source to the corresponding shadow object.

\subsubsection*{Syntax}
\begin{tabular}{ll}
 \verb![F]! & \verb|!$xmp| {\tt reflect} \verb|(| {\it array-name}
 {\openb}, {\it array-name}{\closeb}... \verb|)| {\bsquare} \\
 &\hspace{0.1cm} {\bsquare} {\openb}{\tt width (} {\it reflect-width}
     {\openb}, {\it reflect-width}{\closeb}... {\tt )}{\closeb}
     {\openb}{\tt orthogonal}{\closeb}
     {\openb}{\tt async (} {\it async-id} {\tt )}{\closeb} {\openb}{\tt acc}{\closeb}\\
\verb![C]! & \verb|#pragma xmp| {\tt reflect} \verb|(| {\it array-name}
     {\openb}, {\it array-name}{\closeb}... \verb|)| {\bsquare} \\
 &\hspace{0.1cm} {\bsquare} {\openb}{\tt width (} {\it reflect-width}
     {\openb}, {\it reflect-width}{\closeb}... {\tt )}{\closeb}
     {\openb}{\tt orthogonal}{\closeb}
     {\openb}{\tt async (} {\it async-id} {\tt )}{\closeb} {\openb}{\tt acc}{\closeb}\\
\end{tabular}

\vspace{1em}
where {\it reflect-width} must be one of:
\vspace{1em}

\begin{tabular}{ll}
 \hspace{0.5cm} & {\openb}{\tt /periodic/}{\closeb} {\it int-expr} \\
                & {\openb}{\tt /periodic/}{\closeb} {\it int-expr} : {\it int-expr}
\end{tabular}

\subsubsection*{Description}
When the {\tt acc} clause is specified,
the {\tt reflect} construct updates each of the shadow object of the
array specified by {\it array-name} on accelerator memory with the value of its corresponding
reflection source.

\subsubsection*{Restriction}
\begin{itemize}
 \item When the {\tt acc} clause is specified,
   the arrays specified by the sequence of {\it array-name}'s must be allocated on accelerator memory.
 \item This construct must not appear in {\OACC} {\bf compute region}.
\end{itemize}

\subsubsection*{Example}
\begin{myfigure}
\begin{minipage}{0.45\hsize}
\begin{center}
\begin{XACCFexampleL}
integer :: a(N)
!$xmp template t(N)
!$xmp nodes p(*)
!$xmp distribute t(block) onto p
!$xmp align a(i) with t(i)
!$xmp shadow a(1)
...
!$acc enter data copyin(a)
!$xmp reflect (a) acc
\end{XACCFexampleL}
\end{center}
\end{minipage}
%
\begin{minipage}{0.53\hsize}
\begin{center}
\begin{XACCCexampleR}
int a[N];
#pragma xmp template t[N]
#pragma xmp nodes p[*]
#pragma xmp distribute t[block] onto p
#pragma xmp align a[i] with t[i]
#pragma xmp shadow a[1]
...
#pragma acc enter data copyin(a)
#pragma xmp reflect (a) acc
\end{XACCCexampleR}
\end{center}
\end{minipage}
\caption{Code example in {\tt reflect} construct}\label{code:reflect}
\end{myfigure}

In lines 2-5 of Fig. \ref{code:reflect},
the directives declare {\bf distributed array} {\it a}.
In line 6, 
the {\tt shadow} directive allocates shadow areas of the {\bf distributed array} {\it a}.
In line 8,
the {\tt enter data} directive transfers the {\bf distributed array} {\it a} with the shadow areas from host memory to accelerator memory.
In line 9,
the {\tt reflect} directive updates the shadow areas of the {\bf distributed array} {\it a} on accelerator memory between neighboring {\bf nodes}.

\subsubsection{reflect\_init and reflect\_do Constructs}\label{sec:gmove}
\subsubsection*{Synopsis}
Since the {\tt reflect\_init} construct performs the initialization processes of the {\tt reflect} construct,
the {\tt reflect\_do} construct performs communication of the {\tt reflect} construct.

\subsubsection*{Syntax}
\begin{tabular}{ll}
 \verb![F]! & \verb|!$xmp| {\tt reflect\_init} \verb|(| {\it array-name}
 {\openb}, {\it array-name}{\closeb}... \verb|)| {\bsquare} \\
 &\hspace{0.1cm} {\bsquare} {\openb}{\tt width (} {\it reflect-width}
     {\openb}, {\it reflect-width}{\closeb}... {\tt )}{\closeb}
     {\openb}{\tt orthogonal}{\closeb}
     {\openb}{\tt async (} {\it async-id} {\tt )}{\closeb} {\openb}{\tt acc}{\closeb}\\
\verb![C]! & \verb|#pragma xmp| {\tt reflect\_init} \verb|(| {\it array-name}
     {\openb}, {\it array-name}{\closeb}... \verb|)| {\bsquare} \\
 &\hspace{0.1cm} {\bsquare} {\openb}{\tt width (} {\it reflect-width}
     {\openb}, {\it reflect-width}{\closeb}... {\tt )}{\closeb}
     {\openb}{\tt orthogonal}{\closeb}
     {\openb}{\tt async (} {\it async-id} {\tt )}{\closeb} {\openb}{\tt acc}{\closeb}\\
\end{tabular}

\vspace{1em}
where {\it reflect-width} must be one of:
\vspace{1em}

\begin{tabular}{ll}
 \hspace{0.5cm} & {\openb}{\tt /periodic/}{\closeb} {\it int-expr} \\
                & {\openb}{\tt /periodic/}{\closeb} {\it int-expr} : {\it int-expr}
\end{tabular}

\vspace{1em}

\begin{tabular}{ll}
 \verb![F]! & \verb|!$xmp| {\tt reflect\_do} \verb|(| {\it array-name}
 {\openb}, {\it array-name}{\closeb}... \verb|)| {\openb}{\tt async (} {\it async-id} {\tt )}{\closeb} {\openb}{\tt acc}{\closeb}\\
\verb![C]! & \verb|#pragma xmp| {\tt reflect\_do} \verb|(| {\it array-name}
     {\openb}, {\it array-name}{\closeb}... \verb|)| 
     {\openb}{\tt async (} {\it async-id} {\tt )}{\closeb} {\openb}{\tt acc}{\closeb}\\
\end{tabular}

\subsubsection*{Description}
The {\tt reflect} construct is divided into {\tt reflect\_init} and {\tt reflect\_do} constructs to improve performance like the MPI persistent communication\cite{mpi}.

As a typical example, 
if a {\tt reflect} construct is called repeatedly with the same condition in a loop statement, 
inserting a {\tt reflect\_init} construct before the loop statement 
and replacing the {\tt reflect} construct with a {\tt reflect\_do} construct will improve its performance
because unneeded initialization processes are removed.

\subsubsection*{Restriction}
\begin{itemize}
 \item When the {\tt acc} clause is specified,
   the arrays specified by the sequence of {\it array-name}'s must be allocated on accelerator memory.
 \item These constructs must not appear in {\OACC} {\bf compute region}.
 \item The {\tt reflect\_init} directive must execute before the {\tt reflect\_init} directive executes.
\end{itemize}

\subsubsection*{Example}
\begin{myfigure}
\begin{minipage}{0.45\hsize}
\begin{center}
\begin{XACCFexampleL}
integer :: a(N)
!$xmp template t(N)
!$xmp nodes p(*)
!$xmp distribute t(block) onto p
!$xmp align a(i) with t(i)
!$xmp shadow a(1)
...
!$acc enter data copyin(a)
!$xmp reflect_init (a) acc
...
!$xmp reflect_do (a) acc
\end{XACCFexampleL}
\end{center}
\end{minipage}
%
\begin{minipage}{0.53\hsize}
\begin{center}
\begin{XACCCexampleR}
int a[N];
#pragma xmp template t[N]
#pragma xmp nodes p[*]
#pragma xmp distribute t[block] onto p
#pragma xmp align a[i] with t[i]
#pragma xmp shadow a[1]
...
#pragma acc enter data copyin(a)
#pragma xmp reflect_init (a) acc
...
#pragma xmp reflect_do (a) acc
\end{XACCCexampleR}
\end{center}
\end{minipage}
\caption{Code example in {\tt reflect\_init} and {\tt reflect\_do} constructs}\label{code:reflect_initdo}
\end{myfigure}

In lines 2-5 of Fig. \ref{code:reflect_initdo},
the directives declare {\bf distributed array} {\it a}.
In line 6,
the {\tt shadow} directive allocates shadow areas of the {\bf distributed array} {\it a}.
In line 8,
the {\tt enter data} directive transfers the {\bf distributed array} {\it a} with the shadow areas from host memory to accelerator memory.
In line 9,
the {\tt reflect\_init} directive performs initialization processes for the {\tt reflect\_do} construct which targets the {\bf distributed array} {\it a}.
In line 11,
the {\tt reflect\_do} directive updates the shadow areas of the {\bf distributed array} {\it a} on accelerator memory between neighboring {\bf nodes}
without its initialization processes.

\subsubsection{gmove Construct}\label{sec:gmove}
\subsubsection*{Synopsis}
The {\tt gmove} construct allows an assignment statement,
which may cause communication, to be executed possibly in parallel by
the executing {\bf nodes}.

\subsubsection*{Syntax}
\begin{tabular}{ll}
\verb![F]! & \verb|!$xmp| {\tt gmove} {\openb}{\tt in} $\vert$ {\tt
 out}{\closeb} {\openb}{\tt async (} {\it async-id} {\tt )}{\closeb} {\openb}{\tt acc}{\openb}({\it variable}){\closeb}{\closeb}\\
\verb![C]! & \verb|#pragma xmp| {\tt gmove} {\openb}{\tt in} $\vert$ {\tt
 out}{\closeb} {\openb}{\tt async (} {\it async-id} {\tt )}{\closeb} {\openb}{\tt acc}{\openb}({\it variable}){\closeb}{\closeb}\\
\end{tabular}

\subsubsection*{Description}
\begin{itemize}
 \item When the {\tt acc} clause is specified and the variable is not specified by {\it variable} in the parenthesis,
variables of both sides in the assignment statement on accelerator memory are targeted.
 \item When the {\tt acc} clause is specified and the variable is specified by {\it variable} in the parenthesis,
the specified variable on accelerator memory is targeted, 
and the unspecified variable on host memory is targeted.
\end{itemize}

\subsubsection*{Restriction}
\begin{itemize}
 \item The variables targeted on accelerator memory must be allocated on accelerator memory.
 \item This construct must not appear in {\OACC} {\bf compute region}.
\end{itemize}

\subsubsection*{Example}
\begin{myfigure}
\begin{minipage}{0.45\hsize}
\begin{center}
\begin{XACCFexampleL}
integer :: a(N), b(N)
!$xmp template t(N)
!$xmp nodes p(*)
!$xmp distribute t(block) onto p
!$xmp align a(i) with t(i)
!$xmp align b(i) with t(i)
...
!$acc enter data copyin(a, b)
!$xmp gmove acc
  a(:) = b(:)

!$xmp gmove acc(b)
  a(:) = b(:)
\end{XACCFexampleL}
\end{center}
\end{minipage}
%
\begin{minipage}{0.53\hsize}
\begin{center}
\begin{XACCCexampleR}
int a[N], b[N];
#pragma xmp template t[N]
#pragma xmp nodes p[*]
#pragma xmp distribute t[block] onto p
#pragma xmp align a[i] with t[i]
#pragma xmp align b[i] with t[i]
...
#pragma acc enter data copyin(a, b)
#pragma xmp gmove acc
  a[:] = b[:];

#pragma xmp gmove acc(b)
  a[:] = b[:];
\end{XACCCexampleR}
\end{center}
\end{minipage}
\caption{Code example in {\tt gmove} construct}\label{code:gmove}
\end{myfigure}

In lines 2-6 of Fig. \ref{code:gmove},
the directives declare {\bf distributed arrays} {\it a} and {\it b}.
In line 8,
the {\tt enter data} directive transfers the {\bf distributed arrays} {\it a} and {\it b} from host memory to accelerator memory.
In lines 9-10,
the {\tt gmove} construct copies the whole {\bf distributed array} {\it b} to
that of the {\bf distributed array} {\it a} on accelerator memories.
In lines 12-13,
the {\tt gmove} construct copies the whole {\bf distributed array} {\it b} on accelerator memory to
that of the {\bf distributed array} {\it a} on host memory.

\subsubsection{barrier Construct}\label{sec:barrier}
\subsubsection*{Synopsis}
The {\tt barrier} construct specifies an explicit barrier
at the point at which the construct appears.

\subsubsection*{Syntax}
\begin{tabular}{ll}
\verb![F]! & \verb|!$xmp| {\tt barrier} {\openb}{\tt on} {\it nodes-ref}
 $\vert${\it template-ref}{\closeb} {\openb}{\tt acc}{\closeb}\\
\verb![C]! & \verb|#pragma xmp| {\tt barrier} {\openb}{\tt on} {\it
     nodes-ref} $\vert$ {\it template-ref}{\closeb} {\openb}{\tt acc}{\closeb}\\
\end{tabular}

\subsubsection*{Description}
\begin{itemize}
 \item When the {\tt acc} clause is specified,
the barrier construct blocks until all ongoing asynchronous operations on accelerators are completed.
 \item When the {\tt acc} clause is not specified,
the barrier construct does not guarantee that an ongoing asynchronous operation on accelerator is completed.
\end{itemize}

\subsubsection*{Example}
\begin{myfigure}
\begin{minipage}{0.45\hsize}
\begin{center}
\begin{XACCFexampleL}
!$xmp nodes p(*)
...
!$xmp barrier acc
\end{XACCFexampleL}
\end{center}
\end{minipage}
%
\begin{minipage}{0.53\hsize}
\begin{center}
\begin{XACCCexampleR}
#pragma xmp nodes p[*]
...
#pragma xmp barrier acc
\end{XACCCexampleR}
\end{center}
\end{minipage}
\caption{Code example in {\tt barrier} construct}\label{code:barrier}
\end{myfigure}

In line 1,
the {\tt nodes} directive defines {\tt node set} {\it p}.
In line 3,
the {\tt barrier} directive performs a barrier operation for accelerators on all {\bf node}.

\subsubsection{reduction Construct}\label{sec:reduction}
\subsubsection*{Synopsis}
The {\tt reduction} construct performs a reduction operation among {\bf nodes}.

\subsubsection*{Syntax}
\Syntax{reduction}

\begin{tabular}{ll}
\verb![F]! & \verb|!$xmp| {\tt reduction (} {\it reduction-kind} {\it
  :} {\it variable} {\openb}, {\it variable} {\closeb}... {\tt )}
 {\bsquare} \\
 & \hspace{5cm} {\bsquare} {\openb}{\tt on} {\it node-ref} $\vert$ {\it
     template-ref}{\closeb} {\openb}{\tt async (} {\it async-id} {\tt )}{\closeb} {\openb}{\tt acc}{\closeb}\\
\end{tabular}

\vspace{0.5cm}
where {\it reduction-kind} is one of:

\begin{tabular}{ll}
 \hspace{0.5cm} & {\tt +} \\
 & {\tt *} \\
% & {\tt -} \\
 & {\tt .and.} \\
 & {\tt .or.} \\
 & {\tt .eqv.} \\
 & {\tt .neqv.} \\
 & {\tt max} \\
 & {\tt min} \\
 & {\tt iand} \\
 & {\tt ior} \\
 & {\tt ieor} \\
\end{tabular}

\vspace{0.5cm}

\begin{tabular}{ll}
 \hspace{-\parindent}
 \verb![C]! & \verb|#pragma xmp| {\tt reduction (} {\it reduction-kind} {\it
  :} {\it variable} {\openb}, {\it variable} {\closeb}... {\tt )}
 {\bsquare} \\
 & \hspace{5cm} {\bsquare} {\openb}{\tt on} {\it node-ref} $\vert$ {\it
     template-ref}{\closeb} {\openb}{\tt async (} {\it async-id} {\tt )}{\closeb} {\openb}{\tt acc}{\closeb} \\
\end{tabular}
\vspace{0.5cm}

where {\it reduction-kind} is one of:

\begin{tabular}{ll}
 \hspace{0.5cm} & {\tt +} \\
 & {\tt *} \\
% & {\tt -} \\
 & {\verb|&|} \\
 & {\tt |} \\
 & {\verb|^|} \\
 & {\verb|&&|} \\
 & {\tt ||} \\
 & {\tt max} \\
 & {\tt min} \\
\end{tabular}

\subsubsection*{Description}
When the {\tt acc} clause is specified,
the {\tt reduction} construct performs a type of
reduction operation specified by {\it reduction-kind} for the specified
local variables among the accelerators and 
sets the reduction results to the variables on each of the accelerators.

\subsubsection*{Restriction}
\begin{itemize}
 \item When the {\tt acc} clause is specified,
   the variables specified by the sequence of {\it variable}'s must be allocated on accelerator memory.
 \item This construct must not appear in {\OACC} {\bf compute region}.
\end{itemize}

\subsubsection*{Example}
\begin{myfigure}
\begin{minipage}{0.45\hsize}
\begin{center}
\begin{XACCFexampleL}
integer :: a
!$xmp nodes p(*)
...
!$acc enter data copyin(a)
!$xmp reduction(+:a) acc
\end{XACCFexampleL}
\end{center}
\end{minipage}
%
\begin{minipage}{0.53\hsize}
\begin{center}
\begin{XACCCexampleR}
int a;
#pragma xmp nodes p[*]
...
#pragma acc enter data copyin(a)
#pragma xmp reduction(+:a) acc
\end{XACCCexampleR}
\end{center}
\end{minipage}
\caption{Code example in {\tt reduction} construct}\label{code:reduction}
\end{myfigure}

In line 2,
the {\tt nodes} directive defines {\bf node set} {\it p}.
In line 4,
the {\tt enter data} directive transfers the local variable {\it a} from host memory to accelerator memory.
In line 5,
the {\tt reduction} directive calculates a total value of the variable {\it a} stored on each accelerator
memory in each {\bf node}.

\subsubsection{bcast Construct}\label{sec:bcast}
\subsubsection*{Synopsis}
The {\tt bcast} construct performs broadcast communication from a specified {\bf node}.

\subsubsection*{Syntax}

\begin{tabular}{ll}
 \verb![F]! & \verb|!$xmp| {\tt bcast} \verb|(| {\it variable}
 {\openb}, {\it variable}{\closeb}... \verb|)|
 {\openb}{\tt from} {\it nodes-ref} $\vert$ {\it template-ref}{\closeb}
 {\bsquare} \\
 & \hspace{0.5cm} {\bsquare} {\openb}{\tt from\_device} {\it devices-ref}{\closeb}
 {\openb}{\tt on} {\it nodes-ref}{\closeb} $\vert$ {\it template-ref}{\closeb}
 {\openb}{\tt async (} {\it async-id} {\tt )}{\closeb} {\openb}{\tt acc}{\closeb}\\

 \verb![C]! & \verb|#pragma xmp| {\tt bcast} \verb|(| {\it variable}
 {\openb}, {\it variable}{\closeb}... \verb|)|
 {\openb}{\tt from} {\it nodes-ref} $\vert$ {\it template-ref}{\closeb} {\bsquare} \\
 & \hspace{0.5cm} {\bsquare} {\openb}{\tt from\_device} {\it devices-ref} {\closeb}
 {\openb}{\tt on} {\it nodes-ref} $\vert$ {\it template-ref}{\closeb}
 {\openb}{\tt async (} {\it async-id} {\tt )}{\closeb} {\openb}{\tt acc}{\closeb}\\
\end{tabular}

\subsubsection*{Description}
When the {\tt acc} clause is specified, 
the values of the variables specified by the sequence of {\it variable}'s on accelerator memory
(called {\bf broadcast variables}) are broadcasted from the {\bf node} specified by the {\tt from} clause (called the {\bf source node}) 
to each of the {\bf nodes} in the {\bf node set} specified by the {\tt on} clause.
Moreover, the {\tt from\_device} clause specifies a device which has {\bf broadcast variables}.
Generally, the {\tt from\_device} clause is used when a {\bf devices} directive is used which is described in Section \ref{sec:devices}.
If there is no {\tt from\_device} clause, it is assumed that the default device is specified.
After executing this construct,
the values of the {\bf broadcast variables} become the same as those in the {\bf source node}.


\subsubsection*{Restriction}
\begin{itemize}
 \item When the {\tt acc} clause is specified,
   the variables specified by the sequence of {\it variable}'s must be allocated on accelerator memory.
 \item This construct must not appear in {\OACC} {\bf compute region}.
\end{itemize}

\subsubsection*{Example 1}
\begin{myfigure}
\begin{minipage}{0.45\hsize}
\begin{center}
\begin{XACCFexampleL}
integer :: a
!$xmp nodes p(*)
...
!$acc enter data copyin(a)
!$xmp bcast(a) acc
\end{XACCFexampleL}
\end{center}
\end{minipage}
%
\begin{minipage}{0.53\hsize}
\begin{center}
\begin{XACCCexampleR}
int a;
#pragma xmp nodes p[*]
...
#pragma acc enter data copyin(a)
#pragma xmp bcast(a) acc
\end{XACCCexampleR}
\end{center}
\end{minipage}
\caption{Code example in {\tt bcast} construct}\label{code:bcast}
\end{myfigure}

In line 2,
the {\tt nodes} directive defines {\bf node set} {\it p}.
In line 4,
the {\tt enter data} directive transfers the local variable {\it a} from host memory to accelerator memory.
In line 5,
the {\tt bcast} directive broadcasts the variable {\it a} stored on accelerator memory to all {\it nodes}.

\subsubsection*{Example 2}
\begin{myfigure}
\begin{center}
\begin{XACCFexampleL}
integer :: a
!$xmp nodes p(*)
!$acc devices d(*)
...
!$acc enter data copyin(a) on_device(d)
!$xmp bcast(a) from_device d(1) from p(2) acc
\end{XACCFexampleL}
\end{center}
\begin{center}
\begin{XACCCexampleL}
int a;
#pragma xmp nodes p[*]
#pragma acc devices d[*]
...
#pragma acc enter data copyin(a) on_device(d)
#pragma xmp bcast(a) from_device d[0] from p[1] acc
\end{XACCCexampleL}
\end{center}
\caption{Code example in {\tt bcast} construct with {\tt from\_device} clause}\label{code:bcast2}
\end{myfigure}

In line 2,
the {\tt nodes} directive defines {\bf node set} {\it p}.
In line 3,
the {\tt devices} directive defines device set {\it d}.
In line 5,
the {\tt enter data} directive with {\tt on\_device} clause transfers the local variable {\it a} from host memory to all accelerator memories.
Note that the {\tt on\_device} clause is described in Section \ref{sec:ondevice}.
In line 6,
the {\tt bcast} directive broadcasts the variable {\it a} stored on accelerator memory on {\it d(1)} of {\it p(2)} to all accelerator memories of all {\it nodes}.

\subsubsection{wait\_async Construct}\label{sec:waitasync}
\subsubsection*{Synopsis}
The {\tt wait\_async} construct guarantees asynchronous
communications specified by {\it async-id} are complete.

\subsubsection*{Syntax}
\begin{tabular}{ll}
\verb![F]! & \verb|!$xmp| {\tt wait\_async ( {\it async-id} {\openb},
 {\it async-id} {\closeb}...)} {\openb}{\tt on} {\it nodes-ref} $\vert$
 {\it template-ref}{\closeb} {\openb}{\tt acc}{\closeb}\\
\verb![C]! & \verb|#pragma xmp| {\tt wait\_async ( {\it async-id} {\openb},
 {\it async-id} {\closeb}...)} {\openb}{\tt on} {\it nodes-ref} $\vert$
 {\it template-ref}{\closeb} {\bsquare} \\
& \hspace{13.5cm} {\bsquare} {\openb}{\tt acc}{\closeb}\\
\end{tabular}

\subsubsection*{Description}
When the {\tt acc} clause is specified,
the {\tt wait\_async} construct blocks and therefore
statements following it are not executed until all of the asynchronous
communications that are specified by {\it async-id}'s and issued on the accelerators in
{\bf node set} specified by the {\tt on} clause are complete.

\subsubsection*{Restriction}
This construct must not appear in {\OACC} {\bf compute region}.

\subsubsection*{Example}
\begin{myfigure}
\begin{minipage}{0.45\hsize}
\begin{center}
\begin{XACCFexampleL}
integer :: a
!$xmp nodes p(*)
...
!$acc enter data copyin(a)
!$xmp reduction(+:a) acc async(1)
...
!$xmp wait_async(1) acc
\end{XACCFexampleL}
\end{center}
\end{minipage}
%
\begin{minipage}{0.53\hsize}
\begin{center}
\begin{XACCCexampleR}
int a;
#pragma xmp nodes p[*]
...
#pragma acc enter data copyin(a)
#pragma xmp reduction(+:a) acc async(1)
...
#pragma xmp wait_async(1) acc
\end{XACCCexampleR}
\end{center}
\end{minipage}
\caption{Code example in {\tt wait\_async} construct}\label{code:waitasync}
\end{myfigure}

In line 2,
the {\tt nodes} directive defines {\bf node set} {\it p}.
In line 4,
the {\tt enter data} directive transfers the local variable {\it a} from host memory to accelerator memory.
In line 5,
the {\tt reduction} directive performs asynchronously.
In line 7,
the {\tt wait\_async} construct blocks until the asynchronous reduction operation at line 5 is complete.

\subsection{Coarray Features} \label{sec:coarray}
\subsubsection*{Synopsis}
{\XACC} can perform one-sided communication (put/get operations) for data on accelerator memory using {\bf coarray} features,
which is based on {\XMP} local-view memory model.
A combination of {\bf coarray} syntax and {\tt host\_data} construct enables communication between accelerators.

\subsubsection*{Description}
If {\bf coarrays} appear in {\tt use\_device} clause of any enclosing {\tt host\_data} construct, 
communication targets data on the accelerator side. 
{\bf Coarray} operations on accelerators are synchronized using the same synchronization functions in {\XMP}.

\subsubsection*{Restriction}
\begin{itemize}
 \item Only {\tt declare} directive can declare a {\bf coarray} on accelerator memory.
   For example,
   {\tt enter data} and {\tt copy} directives cannot declare a {\bf coarray} on accelerator memory.
 \item The {\bf coarray} syntax must not appear in {\OACC} {\bf compute region}.
\end{itemize}

\subsubsection*{Example}
\begin{myfigure}
\begin{minipage}{0.45\hsize}
\begin{center}
\begin{XACCFexampleL}
integer :: a(N)[*]
integer :: b(N)
!$acc declare create(a, b)
...
if(this_image() == 1) then
!$acc host_data use_device(a, b)
  a(:)[2] = b(:)

!$acc host_data use_device(a)
  b(:) = a(:)[3]
end if
...
sync all
\end{XACCFexampleL}
\end{center}
\end{minipage}
%
\begin{minipage}{0.53\hsize}
\begin{center}
\begin{XACCCexampleR}
int a[N]:[*];
int b[N];
#pragma acc declare create(a, b)
...
if(xmpc_this_image() == 1){
#pragma acc host_data use_device(a, b)
  a[:]:[2] = b[:];

#pragma acc host_data use_device(a)
  b[:] = a[:]:[3];
}
...
xmp_sync_all(NULL);
\end{XACCCexampleR}
\end{center}
\end{minipage}
\caption{Code example in coarray features}\label{code:coarray}
\end{myfigure}

In line 3 of Fig. \ref{code:coarray},
the {\tt declare} directive declares a {\bf coarray} {\it a} and an array {\it b} on accelerator memory.
In lines 6-7,
{\bf image} 1 performs put operation, where
the whole array {\it b} on accelerator memory in {\bf image} 1 is transferred to the {\bf coarray} {\it a} on accelerator memory in {\bf image} 2.
In lines 9-10,
{\bf image} 1 performs get operation, where
the whole {\bf coarray} {\it a} on accelerator memory in {\bf image} 3 is transferred to the array {\it b} on host memory in {\bf image} 1.
In line 13,
the {\tt sync all} statement in {\XACCF} or the {\tt xmp\_sync\_all} function in {\XACCC} synchronizes all {\bf images} and guarantees completion of ongoing coarray operations.


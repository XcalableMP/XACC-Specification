\chapter{{\OACC} Extension}\label{chap:acc-ex}
This chapter defines an extension of {\OACC} in {\XACC}.
The extension can represent offload works to multiple-accelerators on each {\tt node}.

\section{devices Directive}
\subsection*{Synopsis}
The {\bf devices} directive declares a set of devices.

\subsubsection*{Syntax}
\begin{tabular}{ll}
  \verb![F]! & \verb|!$acc| {\tt devices} {\it devices-decl} {\openb}, {\it devices-decl} {\closeb}...\\
  \verb![C]! & \verb|#pragma acc| {\tt devices} {\it devices-decl} {\openb}, {\it devices-decl} {\closeb}...
\end{tabular}

\vspace{1em}
where {\it device-decl} is one of:
\vspace{1em}

\begin{tabular}{lll}
  \hspace{0.5cm} & & {\it devices-name} \verb|(| {\it devices-spec} \verb|)| \\
  \hspace{0.5cm} & & {\it devices-name} \verb|(| {\it devices-spec} \verb|)| {\openb} {\tt =} {\it predefined-devices-ref} {\closeb} \\
  %{\openb}, {\it devices-spec} {\closeb}...
  \hspace{0.5cm} & \verb![C]! & {\it devices-name} \verb|[| {\it devices-spec} \verb|]| \\
  \hspace{0.5cm} & \verb![C]! & {\it devices-name} \verb|[| {\it devices-spec} \verb|]| {\openb} {\tt =} {\it predefined-devices-ref} {\closeb}
  %{\openb} \verb|[| {\it devices-spec} \verb|]|... {\closeb}
\end{tabular}

\vspace{1em}
and {\it devices-spec} is one of:
\vspace{1em}

\begin{tabular}{ll}
 \hspace{0.5cm} & * \\
                & {\it int-expr}
\end{tabular}

\vspace{1em}
and {\it predefined-devices-ref} is one of:
\vspace{1em}

\begin{tabular}{ll}
  \hspace{0.5cm} & {\it device-type-name} \verb|(| * $\vert$ {\it int-expr} $\vert$ {\it int-expr} : {\it int-expr}\verb|)| \\
                 & {\it device-type-name} \verb|[| * $\vert$ {\it int-expr} $\vert$ {\it int-expr} : {\it int-expr}\verb|]|
%                & {\it device-type-name} \verb|(| {\it int-expr} \verb|)|\\
%                & {\it device-type-name} \verb|(| {\it int-expr} : {\it int-expr} \verb|)|
\end{tabular}

\subsubsection*{Description}
The {\bf device} directive declares a device array that corresponds to a device set.

The first and third forms are used to declare a device array that corresponds to the entire device set.
The second and fourth forms are used to declare a device array, each device of which is
assigned to a device of the device set is specified by {\it predefined-devices-ref} at the corresponding position.

%% In the first and second forms which use parentheses,
%% the corresponding position is Fortran's array element order, as if the device set were a one-dimensional device array.
%% In the third and fourth forms which use square brackets,
%% the corresponding position is C's array element order, as if the device set were a one-dimensional device array.

\subsubsection*{Restriction}
\begin{itemize}
\item {\it devices-name} must not conflict with any other local name in
      the same scoping unit.
 %% \item When the {\bf acc} clause is specified,
 %%   the arrays specified by the sequence of {\it array-name}'s must be allocated on accelerator memory.
 \item This construct must not appear in {\OACC} {\tt compute region}.
\end{itemize}

\subsubsection*{Example}
The following are examples of the devices declaration. The device array {\tt d} corresponds to the entire default device set and the device array {\tt e} is a subset of the predefined device array {\tt nvidia}. The program must be executed by a node which has four or more NVIDIA accelerator devices.
%
\begin{myfigure}
\begin{minipage}{0.45\hsize}
\begin{center}
\begin{XACCFexampleL}
!$acc devices d(*)
!$acc devices e(2) = nvidia(3:4)  
\end{XACCFexampleL}
\end{center}
\end{minipage}
%
\begin{minipage}{0.53\hsize}
\begin{center}
\begin{XACCCexampleR}
#pragma acc devices d[*]
#pragma acc devices e[2] = nvidia[2:2]
\end{XACCCexampleR}
\end{center}
\end{minipage}
\caption{Code example in {\XACC} {\bf devices} directive}\label{code:devices}
\end{myfigure}



\subsection{Device Reference}
\subsection*{Synopsis}
The device reference is used to reference a device set.

\subsubsection*{Syntax}
\begin{tabular}{llll}
             & {\it devices-ref} & {\bf is} & {\it devices-name} {\openb}\verb|(| {\it devices-subscript} \verb|)|{\closeb}\\
  \verb![C]! & {\it devices-ref} & {\bf is} & {\it devices-name} {\openb}\verb|[| {\it devices-subscript} \verb|]|{\closeb}
\end{tabular}

\vspace{1em}
where {\it devices-subscript} must be one of:
\vspace{1em}

\begin{tabular}{ll}
 \hspace{0.5cm} & {\it int-expr} \\
                & {\it triplet}
\end{tabular}

\subsubsection*{Description}
A device reference by {\it devices-name} represents a device set corresponding to the device array specified by the name or its subarray.
%% A device reference by ``*'' represents the executing device set.

\subsubsection*{Example}
Assume that {\tt d} is the name of a device array.
\begin{itemize}
\item To specify a device set to which the declared device array corresponds,\\

\begin{myfigure}
\begin{minipage}{0.40\hsize}
\begin{center}
\begin{XACCFexampleL}
!$acc devices e(1) = d(1)
!$acc devices f(3) = d(2:4)
\end{XACCFexampleL}
\end{center}
\end{minipage}
%
\begin{minipage}{0.45\hsize}
\begin{center}
\begin{XACCCexampleR}
#pragma acc devices e[1] = d[0]
#pragma acc devices f[1] = d[1:3]
\end{XACCCexampleR}
\end{center}
\end{minipage}
\caption{Code example in {\XACC} device reference}\label{code:device_ref}
\end{myfigure}

\item ...

\end{itemize}



\section{Data Mapping Clauses}
\subsection{layout Clause}

\subsection{shadow Clause}



\section{Communication on Accelerators}
\subsection{barrier\_device Construct}
\subsection*{Synopsis}
The {\tt barrier\_device} construct specifies an explicit barrier among devices at the point which the construct appears.

\subsubsection*{Syntax}
\begin{tabular}{ll}
  \verb![F]! & \verb|!$acc barrier_device|       {\openb}\verb|on_device(| {\it devices-ref} \verb|)|{\closeb}\\
  \verb![C]! & \verb|#pragma acc barrier_device| {\openb}\verb|on_device(| {\it devices-ref} \verb|)|{\closeb}
\end{tabular}

\subsubsection*{Description}
The barrier\_device construct blocks accelerator devices until all outgoing asynchronous operations on them are completed regardless of the host operations.
The construct is performed among the device set specified by the {\tt on\_device} clause.
If no {\tt on\_device} clause is specified, then it is assumed that the default device set is specified in it.

\subsubsection*{Restriction}
\begin{itemize}
\item This construct must not appear in {\OACC} {\tt compute region}.
\end{itemize}

\subsubsection*{Example}
The following are examples of the barrier\_devices construct.
In line 1--2, the {\tt devices} directives define device set {\tt d} and {\tt e}.
In line 4--5, the first {\tt barrier\_device} construct performs a barrier operation for all devices, 
and the second one performs a barrier operation for devices in the device set {\tt e}.
%
\begin{myfigure}
\begin{minipage}{0.45\hsize}
\begin{center}
\begin{XACCFexampleL}
!$acc devices d(*)
!$acc devices e(2) = d(1:2)
...  
!$acc barrier_device
!$acc barrier_device on_device(e)  
\end{XACCFexampleL}
\end{center}
\end{minipage}
%
\begin{minipage}{0.53\hsize}
\begin{center}
\begin{XACCCexampleR}
#pragma acc devices d[*]
#pragma acc devices e[2] = d[0:2]
...
#pragma acc barrier_device
#pragma acc barrier_device on_device(e)
\end{XACCCexampleR}
\end{center}
\end{minipage}
\caption{Code example in {\XACC} {\bf barrier\_device} construct}\label{code:barrier_device}
\end{myfigure}
